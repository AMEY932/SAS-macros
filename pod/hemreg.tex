
% translated by spod2latex from hemreg.pod on Thu Oct 12 15:54:14 EDT 2006
\subsection{The \macro{HEMREG}: Extract H and E matrices for multivariate regression}\label{mac:hemreg}
 \index{\macro{hemreg}}

The \macro{HEMREG} extracts hypothesis (H) and error (E) matrices for an
overall test in a multivariate regression analysis, in a form similar to
that provided by the \texttt{OUTSTAT=} option with \PROC{GLM}.  This is typically
used with the \macro{HEPLOT}, or the \macro{MPOWER} for MMRA.

\subsubsection*{Method}

For a multivariate regression analysis, using

\begin{listing}
 proc glm outstat=stats;
    model y1 y2 y3 = x1-x5;
\end{listing}

\PROC{GLM} will produce 5 separate 3x3, 1 df SSCP matrices for the separate
predictors X1-X5, in the \texttt{OUTSTAT=} data set, but no SSCP matrix for
the overall multivariate test. The \macro{HEMREG} uses \PROC{REG} instead,
obtains the HypothesisSSCP and ErrorSSCP tables using ODS, and massages
these into the same format used by \PROC{GLM}.

\subsubsection*{Usage}

The \macro{HEMREG} is defined with keyword parameters.  The \texttt{Y=} and
\texttt{X=} parameters are required.  One or more overall hypotheses involving
subsets of the \texttt{X=} variables may be specified with the \texttt{MTEST=} parameter.
The arguments may be listed within parentheses in any order, separated
by commas. For example:

\begin{listing}
 %hemreg(y=SAT PPVT RAVEN, x=N S NS NA SS);
 %hemreg(y=SAT PPVT RAVEN, x=N S NS NA SS, 
      mtest=%str(N,S,NS), hyp=N:S:NS);
\end{listing}
\subsubsection*{Parameters}
\begin{proglist}

\item[DATA=] Name of input dataset. \default{DATA=_LAST_}

\item[Y=] List of response variables.  Must be an explicit,
blank-seaparated list of variable names, and all variables
must be numeric.

\item[X=] List of predictor variables.  Must be an explicit,
blank-seaparated list of variable names, and all variables
must be numeric.

\item[HYP=] Name for each overall hypothesis tested, corresponding to
the test(s) specified in the \texttt{MTEST=} parameter (to be used as 
the \texttt{EFFECT=} parameter in the \macro{HEPLOT}).  \default{HYP=H1}

\item[MTEST=] If \texttt{MTEST=} is not specified (the default), a multivariate test
of all \texttt{X=} predictors is carried out, giving an overall H matrix.
Otherwise, \texttt{MTEST=} can specify one or more multivariate tests of
subsets of the predictors, separated by '/', where the variables
within each subset are separated by ','. In this case, the embedded
','s must be protected by surrounting the parameter value in
\texttt{\%str()}.  For example,

\begin{listing}
 MTEST = %str(group / x1, x2, x3 / x4, x5)
\end{listing}

In this case you might specify \texttt{HYP=Group X1:X3 X4:X5} to name
the H matrices.

\item[SS=] Type of SSCP matrices to compute: Either SS1 or SS2, corresponding
to sequential and partial SS computed by \PROC{REG}. If \texttt{SS=SS2},
the \verb|_TYPE_| variable in the output data set is changed to
\verb|_TYPE_='SS3'|  to conform with \PROC{GLM}. \default{SS=SS2}

\item[OUT=] The name of output HE dataset. \default{OUT=HE}

\end{proglist}
