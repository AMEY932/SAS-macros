% LaTeX document produced by pod2latex from "panels.pod".
% The followings need be defined in the preamble of this document:
%\def\underscore{\leavevmode\kern.04em\vbox{\hrule width 0.4em height 0.3pt}}
%\setlength{\parindent}{0pt}

\section{PANELS}%


\subsection*{Description}
The \macro{PANELS} constructs a template in which to replay a series of
graphs, assumed all the same size, in a rectangular array of R
rows and C columns. By default, the panels are displayed
left-to-right across rows,  starting either from the top
(\texttt{ORDER=DOWN}) or bottom (\texttt{ORDER=UP}).  If the number of rows and
columns are unequal, the aspect ratio of individual panels can
be maintained by setting equate=Y.  It is assumed that all the
plots have already been created, and stored in a graphics catalog
(the default, WORK.GSEG is used automatically by SAS/GRAPH
procedures).

For interactive use within the SAS Session Manager you should be
aware that all plots are stored cumulatively in the graphics
catalog throughout your session, unless explicitly changed with
the \texttt{GOUT=} option in graphics procedures or macros  To create multiple
panelled plots you can use the \texttt{FIRST=} and \texttt{LAST=} parameters or a \texttt{REPLAY=}
list to specify which plots are used in a given call.

\subsection*{Usage}
Call the \macro{PANELS} after the steps which create the graphs in the
graphics catalog.  The \macro{GDISPLA} may be used to suppress the
display of the original full-sized graphs.
The \texttt{ROWS=} and \texttt{COLS=} parameters must be specified.
\begin{listing}
  goptions hsize=7in vsize=5in;
  %gdispla(OFF);
  proc gplot data=mydata;
     plot y * x = group;
     by sex;
\end{listing}
\begin{listing}
  %gdispla(ON);
  %panels(rows=1, cols=2);
\end{listing}

\subsubsection*{Parameters}%
\index{Parameters}

\begin{proglist}

\item[ROWS=]
\item[COLS=]
The \texttt{ROWS=} and \texttt{COLS=} arguments are required, and specify the
size of the array of plots to be displayed. These are the only
required arguments.

\item[PLOTS=]
If there are fewer than \texttt{\&ROWS*\&COLS} plots, specify the number
as the \texttt{PLOTS=} argument. Optionally, there can be an additional
plot, which is displayed (as a GSLIDE title, for example) in
the top nn\% of the display, as specified by the \texttt{TOP=} argument.

\item[TOP=]
If TOP=nn is specified, the top nn\% of the display is reserved
for one additional panel (of width 100\%), to serve as the plot
title or annotation.

\item[ORDER=]
The \texttt{ORDER=} argument specifies the order of the panels in the
\texttt{REPLAY=} list, when \texttt{REPLAY=} is not specified. 
Typically, the panels are displayed across the columns.

\texttt{ORDER=UP} means that the panels in the bottom row are
are drawn first, and  numbered 1, 2, ..., \texttt{\&COLS}. \texttt{ORDER=DOWN} means
that the panels in the top row are drawn first, numbered 1, 2, ..., \texttt{\&COLS}.

If you add the keyword \texttt{BYROWS} to \texttt{ORDER=}, the panels are
displayed up or down the rows.  For example, when \texttt{ROWS=3}, \texttt{COLS=5},
\texttt{ORDER=DOWN BYROWS} generates the \texttt{REPLAY=} list as,
\begin{listing}
       replay=1:1  2:4  3:7  4:10  5:13
              6:2  7:5  8:8  9:11 10:14
             11:3 12:6 13:9 14:12 15:15
\end{listing}

\item[EQUATE=]
The \texttt{EQUATE=} argument determines if the size of the panels is adjusted
so that the aspect ratio of the plots is preserved.  If \texttt{EQUATE=Y},
the size of each plot is adjusted to the maximum of \texttt{\&ROWS} and \texttt{\&COLS}.
This is usually desired, as long as the graphic options HSIZE and
VSIZE are the same when the plots are replayed in the panels
template as when they were originally generated.  The default is
\texttt{EQUATE=Y}.

\item[REPLAY=]
The \texttt{REPLAY=} argument specifies the list of plots to be replayed
in the constructed template, in one of the forms used with the
\PROC{GREPLAY} REPLAY statement, for example, \texttt{REPLAY=1:1 2:3 3:2 4:4} 
or \texttt{REPLAY=1:plot1 2:plot3 3:plot2 4:plot4}

\item[TEMPLATE=]
The name of the template constructed to display the plots. The
default is \texttt{TEMPLATE=PANEL\&ROWS.\&COLS}.

\item[TC=]
The name of the template catalog used to store the template.
You may use a two-part SAS data set name to save the template
permanently.

\item[FIRST=]
By default, the \texttt{REPLAY=} argument is constructed to replay plot
i in panel i.  If the \texttt{REPLAY=} argument is not specified, you can
override this default assignment by specifying \texttt{FIRST=} the sequential
number of the first graph in the graphics catalog to plot (default:
FIRST=1), where:                       

\begin{proglist}

\item[$>$0]
A positive integer means the absolute number of the first graph in the input catalog to be
replayed.  For example, FIRST=3 starts with the third graph.
      
=item $<$1 

An integer less than 1 means the number of first graph relative to last
graph in the input catalog (i.e. FIRST=0 means 
last graph only, FIRST=-1 means the first is the one before last, etc.) 

\end{proglist}

\item[LAST=]
The LAST= parameter may be used to specify the number of the last graph
in the input graphics catalog to be replayed.  The default is LAST=0,
which refers to the last plot in the graphics catalog.  The LAST= value
is interpreted as follows:

\begin{proglist}

\item[$>$0]
A positive integer means the absolute number of last graph in the input catalog to be
replayed.  For example, LAST=4 ends with the fourth graph.

\item[$<$1]
An integer less than 1 means the number of last graph relative to last
graph in the input catalog (i.e. LAST=0 means 
last graph only, LAST=-1 means to end with the one before last, etc.) 

\end{proglist}

\item[GIN=]
GIN specifies the name of the input graphics catalog, from which the
plots to be replayed are taken. The default is \texttt{GIN=WORK.GSEG}.

\item[GOUT=]
\texttt{GOUT=} specifies the name of the graphics catalog in which the panelled
plot is stored. The default is \texttt{GOUT=WORK.GSEG}.

\end{proglist}

