\section{Description\label{Description}\index{Description}}


Given a character or discrete numerical variable, the DUMMY macro creates
dummy (0/1) variables to represent the levels of the original variable
in a regression model.  If the original variable has c levels, (c-1)
new variables are produced (or c variables, if \texttt{FULLRANK=0})

\section{Usage\label{Usage}\index{Usage}}


The DUMMY macro takes the following named parameters.  The arguments
may be listed within parentheses in any order, separated by commas.
For example:

\begin{verbatim}
  %dummy(var=sex group,prefix=);
\end{verbatim}
\subsection*{Parameters\label{Parameters}\index{Parameters}}
\begin{description}

\item[DATA=] \mbox{}

The name of the input dataset.  If not specified, the most
 recently created dataset is used.


\item[OUT=] \mbox{}

The name of the output dataset.  If not specified, the new
variables are appended to the input dataset.


\item[VAR=] \mbox{}

The name(s) of the input variable(s) to be dummy coded.  Must be
specified.  The variable(s) can be character or numeric.


\item[PREFIX=] \mbox{}

Prefix(s) used to create the names of dummy variables.  The
default is 'D\_'.


\item[NAME=] \mbox{}

If \texttt{NAME=VAL}, the dummy variables are named by appending the
value of the \texttt{VAR=} variable to the prefix.  Otherwise, the
dummy variables are named by appending numbers, 1, 2, ...
to the prefix.  The resulting name must be 8 characters or
less.


\item[BASE=] \mbox{}

Indicates the level of the baseline category, which is given
values of 0 on all the dummy variables.  \texttt{BASE=\_FIRST\_ }
specifies that the lowest value of the \texttt{VAR=} variable is the
baseline group; \texttt{BASE=\_LAST\_ }specifies the highest value of
the variable.  Otherwise, you can specify BASE=value to
make a different value the baseline group.


\item[FULLRANK=] \mbox{}

0/1, where 1 indicates that the indicator for the BASE category
          is eliminated.

\end{description}
\subsection*{Example\label{Example}\index{Example}}


With the input data set,

\begin{verbatim}
 data test;
   input y group $ @@;
  cards;
  10  A  12  A   13  A   18  B  19  B  16  C  21  C  19  C  
  ;
\end{verbatim}


The macro statement:

\begin{verbatim}
   %dummy ( data = test, var = group) ;
\end{verbatim}


produces two new variables, D\_A and D\_B.  Group C is the baseline
category (corresponding to \texttt{BASE=\_LAST\_})

\begin{verbatim}
     OBS     Y    GROUP    D_A    D_B
        1     10      A       1      0
        2     12      A       1      0
        3     13      A       1      0
        4     18      B       0      1
        5     19      B       0      1
        6     16      C       0      0
        7     21      C       0      0
        8     19      C       0      0
\end{verbatim}
