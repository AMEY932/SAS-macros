
% translated by spod2latex from canplot.pod on Fri Sep 29 16:37:26 EDT 2006
\subsection{The \macro{CANPLOT}: Canonical discriminant structure plot.}\label{mac:canplot}
 \index{\macro{canplot}}

The \macro{CANPLOT} constructs a canonical discriminant structure
plot. The plot shows class means on the two largest canonical
variables, confidence circles for those means, and variable vectors
showing the correlations of variables with the canonical variates.

\subsubsection*{Method}

Discriminant scores and coefficients are extracted from \PROC{CANDISC}
and plotted.

Other designs may be handled either by (a) coding factor combinations
'interactively', so, e.g., the combinations of A*B are represented by
a GROUP variable, or (b) by applying the method to adjusted response
vectors (residuals) with some other predictor (class or continuous)
partialled out.  The latter method is equivalent to analysis of
the residuals from an initial \PROC{GLM} step, with the effects to be
controlled or adjusted for as predictors.

e.g., to examine Treatment, controlling for Block and Sex,

\begin{listing}
  proc glm data=..;
   model Y1-Y5 = block sex;
   output out=resids
      r=E1-E5;
 %canplot(data=resids, var=E1-E5, class=Treat, ... );
\end{listing}
\subsubsection*{Usage}

The \macro{CANPLOT} is defined with keyword parameters.
Values must be supplied for the \texttt{CLASS=} and \texttt{VAR=} parameters.
The arguments may be listed within parentheses in any order, separated
by commas. For example:

\begin{listing}
 %canplot(data=inputdataset, var=predictors, class=groupvariable...,);
\end{listing}

The interpretation of the angles betweeen variable vectors
relies on the units for the horizontal and vertical axes being made
equal (so that 1 data unit measures the same length on both axes.
The axes should be equated either by using the GOPTIONS \texttt{HSIZE= VSIZE=}
options, or using the macro \texttt{HAXIS=} and \texttt{VAXIS=} parameters
and AXIS statements which specify the \texttt{LENGTH=} value for both
axes. The current version now uses the \macro{EQUATE} if the 
\texttt{HAXIS=} and \texttt{VAXIS=} arguments are not supplied.

\subsubsection*{Parameters}
\begin{proglist}

\item[DATA=] Name of data set to analyze. \default{DATA=_LAST_}

\item[CLASS=] Name of one class variable, defining the groups to be
discriminated.

\item[VAR=] List of classification variables

\item[ID=] Observation ID variable, used to label observations in the plot.

\item[VARLAB=] How to label variables? \verb|_NAME_| or \verb|_LABEL_|. \default{VARLAB=_NAME_}

\item[DIM=] Number of canonical dimensions to be extracted. \default{DIM=2}

\item[SCALE=] Scale factor for variable vectors in plot. The variable 
vectors are multiplied by the \texttt{SCALE=} value, which should
be specified (perhaps by trial and error) to make the vectors and
observations fill the same plot region. \default{SCALE=4}

\item[CONF=] Confidence probability for canonical means, determining the ra. \default{CONF=.99}

\item[OUT=] Output data set containing discrim scores. \default{OUT=_DSCORE_}

\item[OUTVAR=] Output data set containing coefficients. \default{OUTVAR=_COEF_}

\item[ANNO=] Output data set containing annotations. \default{ANNO=_DANNO_}

\item[ANNOADD=] Additional annotations to add to the plot.  Can include
'MEAN' and/or 'GPLABEL' and/or the name(s) of additional
input annotate data sets. \default{ANNOADD=MEAN}

\item[PLOT=] YES (or NO to suppress plot) \default{PLOT=YES}

\item[HAXIS=] The name of an optional AXIS statement for 
the horizontal axis. The \texttt{HAXIS=} and \texttt{VAXIS=} arguments may be used
to equate the axes in the plot so that the units are the same
on the horizontal and vertical axes.  If neither \texttt{HAXIS=} nor
\texttt{VAXIS=} are supplied, the \macro{EQUATE} is called to generate
axis statements.

\item[VAXIS=] The name of an optional AXIS statement for 
the  vertical axis.

\item[INC=] X, Y axis tick increments, in data units. \default{INC=1 1}

\item[XEXTRA=] \# of extra X axis tick marks on the left and right.  Use this to
extend the axis range. \default{XEXTRA=0 0}

\item[YEXTRA=] \# of extra Y axis tick marks on the bottom and top. \default{YEXTRA=0 0}

\item[LEGEND=] Name of a LEGEND statement to specify legend for groups.  Use
\texttt{LEGEND=NONE }to suppress the legend (perhaps with \texttt{ANNOADD=GPLABEL}
to plot group labels near the means).

\item[HSYM=] Height of plot symbols. \default{HSYM=1.2}

\item[HID=] Height of ID labels. \default{HID=1.4}

\item[IDCOLOR=] Color of ID labels

\item[HTEXT=] Height of variable and group labels. \default{HTEXT=1.5}

\item[CANX=] Horizontal axis of plot. \default{CANX=CAN1}

\item[CANY=] Vertical axis of plot. \default{CANY=CAN2}

\item[DIMLAB=] Dimension label prefix. \default{DIMLAB=Canonical Dimension}

\item[COLORS=] List of colors to be used for groups (levels of the \texttt{CLASS=} variable).
The values listed are recycled as needed for the number of groups.
\newline
\default{COLORS=RED GREEN BLUE BLACK PURPLE BROWN ORANGE YELLOW}

\item[SYMBOLS=] List of symbols to be used for the observations within the groups,
recycled as needed.
\default{SYMBOLS=dot circle triangle square star - : \$  =}

\item[LINES=] List of line style numbers used for the confidence circles.
\newline
\default{LINES=20 20 20 20 20 20 20}

\item[NAME=] Name for graphic catalog entry. \default{NAME=CANPLOT}

\item[GOUT=] The name of the graphics catalog. \default{GOUT=GSEG}

\end{proglist}
