
% translated by spod2latex from boxcox.pod on Wed Apr 13 12:48:14 EDT 2005
\section{The \macro{BOXCOX}: Power transformations by Box-Cox method}\label{mac:boxcox}
 \index{\macro{boxcox}}

The \macro{BOXCOX} computes the maximum likelihood power transformation
(or folded-power transformation) of the response variable in a linear
regression with zero or more predictors.  It produces a variety of plots,
including a display of the maximum likelihood solution, t-values for
effects in the model, and an influence plot of the observations in
determining the power.  The optimal transformation of the response
variable is returned in an output dataset.

If the response variable is bounded on a closed interval, [0, b],
the \texttt{FOLD=} parameter may be used to obtain analogous folded-power
transformations.  For example, use \texttt{FOLD=100 }when the response
variable is a percentage on the interval [0, 100].

\subsection*{Method}

The program uses transforms the response to all powers from
the \texttt{LOPOWER=} value to the \texttt{HIPOWER=} value, and fits a regression
model for each, extracting values to an output dataset from
which the plots are drawn.

\subsection*{Usage}

The \macro{BOXCOX} is defined with keyword parameters.  The \texttt{RESP=}
parameter is required.
The arguments may be listed within parentheses in any order, separated
by commas. For example:

\begin{listing}
 %boxcox(data=duncan, resp=prestige, model=income educ, id=job);
\end{listing}
\subsubsection*{Parameters}
\begin{proglist}

\item[RESP=] Name of the response variable for analysis

\item[MODEL=] The independent variables in the regression, i.e., the 
terms on the right side of the = sign in the MODEL statement 
for \PROC{REG}.  May be empty, in which case the response
is transformed toward a normal distribution.

\item[ID=] Name of an ID variable for observations used as labels in
the INFL plot.

\item[DATA=] The name of the data set holding the response and predictor
variables.  \default{DATA=_LAST_}

\item[OUT=] Output dataset with transformed resp. Contains all original
variables, with the transformed response replacing the 
original variable. \default{OUT=_DATA_}

\item[OUTPLOT=] Output dataset for plot of powers Contains \verb|_RMSE_|, and 
t-values for each effect in the model, with one observation
for each power value tried. \default{OUTPLOT=_PLOT_}

\item[PPLOT=] Printer plots: One or more of RMSE, LOGL, EFFECT, and INFL. \default{PPLOT=NONE}

\item[GPLOT=] Graphic plots: One or more of RMSE, LOGL, EFFECT, and INFL. \default{GPLOT=RMSE EFFECT INFL}

\item[ADD=] Additive constant for response, used to avoid problems when
the response can be negative or zero.

\item[FOLD=] Upper bound on the response when the response
is bounded above and below.. 
If FOLD$>$0 is specified, folded power transformations 
are computed. E.g., for a response which is a proportion, 
specify \texttt{FOLD=1}; for a percentage, specify \texttt{FOLD=100}.
\default{FOLD=0}

\item[LOPOWER=] Low value for power. \default{LOPOWER=-2}

\item[HIPOWER=] High value for power. \default{HIPOWER=2}

\item[NPOWER=] Number of power values in interval. \default{NPOWER=21}

\item[CONF=] Confidence coefficient of CI on power. \default{CONF=0.95}
